\documentclass{article}




\title{Statement of purpose}
\author{Jingyi Xu}
\date{October 2018}

\begin{document}
\begin{Large}
\maketitle

\par{Introduction}
\par{When I was in my junior year, I joined the XXX lab led by Prof. Jufeng Yang and worked mainly on visual sentiment analysis, which aims at understanding the sentiment implied in images. The current methods mainly use large-scale web dataset to train a convolutional neural network(CNN). I tried to improve the performance by utilizing the facial regions within an image since they always convey more significant sentiment information compared to other regions.  Specifically, I first used OpenFace, an open source face recognition tool to implement face detection and alignment on the dataset. Then I fed the detected face into a pre-trained emotion recognition CNN model and extracted the representation layer’s output as facial region features. I concatenated it with the output of the original CNN model, which served as the global feature. The concatenated feature vectors are finally used to train a support vector machine(SVM). I tested the SVM classifier and the performance improved by 1.6 percent compared to the baseline.}
\par{I once read a paper for solving pixel-level problems by regressing pixels into a hyper-spherical space so that pixels from the same group have high cosine similarity while those from different groups have similarity below a specific margin.  I was thinking whether the same idea can be applied to face verification tasks in which intra-class similarity and inter-class variance are especially critical. Inspired by the idea, I tried different methods using Pytorch library. First I calculated the pairwise angular distance of samples from different classes within a batch and tried to maximize it to enlarge inter-class variance. However, the loss did not converge. Later, I pre-computed a ‘center’ for each class and include the angular distance between samples and their corresponding center into loss function in order to increase intra-class similarity. Although the loss converges, the accuracy does not improve compared to the original method. After several such failed attempts, I finally proposed a regularization term, named ‘exclusive regularization’, to constrain the parameters of the classification layer and further separate samples of different identites. The result surprisingly outperforms the state-of-the-art. I wrote a paper about my experiments and submmited it to CVPR 2019.}
\par{In the summer, I served as a research assistant in the University of Notre Dame under the supervision of Prof. Zhiyong Zhang. The objective of my project is to explore the factors which have an impact on the ‘importance’ of R packages. First I collected data about the dependency relationship among R packages and constructed a dependency graph. To rank the packages, I studied graph theory and methods for network analysis. The packages can be ranked according to their degree centrality, closeness centrality or betweenness centrality . Then I trained a model using some other attributes of the packages, such as the authors and the published date as training data and the ranking result as labels. During the analysis of the graph, I realized the importance of visual representation of the data. So I built a web application which can generate an interactive graph based on their dependency. Nodes and edges can be displayed in various layouts, colors and sizes. The application is running on a server and can be used to analyze data collected for other practical purposes. Through this experience, I gained some basic mathematics background knowledge in statistics, which lays a solid foundation for understanding machine learning algorithms.}
\par{To further pursue my interests in computer vision, I participated in a face recognition project in Panasonic R\&D Center, Singapore.  
}
\par{As for my future career goals, I hope to work as a research scientist in Computer Labs or be a professor in University. I aspire to apply the statistical learning theory to solve computer vision problems. Mathematical theorems are intrinsically fun for me and I would like to explore how they can be used on vision tasks to benefit users and researchers in the society. Via pursuing a Ph.D. degree, I can improve my programming skills, enhance my thinking ability, and have access to the cutting-edge technologies in the area, which prepares me well for realizing my future goal. 
}
\par{For these reasons, the Computer Science Department of Stony Brook University is especially attractive to me. The objectives of the program, which include training students’ ability to engage in study and advance professionally, are in line with mine. I am particularly impressed by a variety of projects carried out by Professor Dimitris Samaras.  The paper titled ‘Shadow detection with conditional generative adversarial networks’ , which introduces scGAN to solve the challenging problem of shadow detection, is novel and fascinating. In another work ‘ConvNets with smooth adaptive activation functions for regression’, SAAF is proposed to handle regression tasks and avoid overfitting problems.}

\end{Large}
\end{document}
